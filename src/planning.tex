\section{OUTLINE: Ethical Analysis Planning}

\subsection{Kantian Framework}
From this approach, it is possible to evaluate the following 2 things:

\begin{itemize}
	\item{The axiom of access to information is contradicting itself as only one party has control of the information.}
	\item{It also treats others as a mean to an end by invading their privacy for matters of national security, this could be seen as an Imperfect Duty vs a Perfect Duty.}
\end{itemize}

This points towards answering our question with a no with a bit of a grey area depending on how we define the national security as a duty.

\subsection{Act Utilitarianism}
This is a very strong contender for a yes, this is due to the fact that the government can equate the harm of the nation as a major loss, however, with the spyware, it is able to reduce the overall harm by ensuring the security of it's citizens. This is one way this framework defends this, on the other hand, if we consider the emotional distress and oppression of speech that this enables the government to do, that is also another form of harm which is not none. This means that based on the government and the security of the country, this could either be a yes or a no. As in some countries, said spyware is not needed while other countries which face stronger crime might benefit from overall spyware.

\subsection{Social Contract Theory}
The difficulty in trying to analyze the situation with this framework is the fact that this doesn't put in place a veil of ignorance as the people who decide to acquire the spyware are already in their position of power (See the case with the president of Mexico). This also forgets to take into account the fact that those in power are not always doing the best thing for the less privileged (see the case of the women's rights activist getting spied on).

What this allows us to conclude is that the usage of spyware to spy on the citizens would only be ethically sound if the government were to disclose the usage and allow people to reject it (which kind of defeats its purpose as a cyber weapon).

\section{Conclusion}
As seen in the ethical analysis above, the answer leans heavily towards a no. However, there are benefits to using certain tools, but this requires the assumption that the government is mainly there to serve the people of its nation.

\section{Possible Sources}
\begin{itemize}
	\item{WeChat and the way it is used to censor the opposition}
	\item{Articles on the assassination of various activists and journalists}
	\item{Emotional distress caused by spyware (in the case of countries that were using Pegasus)}
	\item{Statistics with government trust per country}
	\item{Immigration procedures and the amount of information they gather on the individual}
\end{itemize}
