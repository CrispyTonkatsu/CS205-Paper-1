\section{Stakeholders}
There were many people involved in this situation, they all have different levels of control over the situation and how to approach it:

\begin{itemize}
	\item{Civilians and journalists, they don't have any power to do anything about the usage of the spyware.}
	\item{Software developers, they have to deliver their promise of security and privacy to their users.}
	\item{Governments, they have the duty to provide security and safety to their citizens and have control over the usage of Pegasus.}
	\item{NSO Group, they have the control on making it, how it works and selling it to clients.}
\end{itemize}

\section{Uses of Pegasus}
Given the concealed nature of this spyware, it took a long time for it to become public knowledge that this was a tool being used by governments. During this time, there were several cases of political persecution and assassination of several individuals.

\subsection{Jamal Khashoggi}
Jamal Khashoggi was a journalist that covered major historical events such as the Soviet invasion of Afghanistan and the rise of Osama Bin Laden. He was close to the royal family as an adviser for the government. \cite{JamalBackground}

However, his close ties to the Saudi Arabian government were damaged later on in his career when he became more openly critical of the regime and more specifically the increase in political persecution and lack of change in the country's politics. \cite{JamalWashingtonPost}

On October 2 2018, Jamal Khashoggi had gone to the Saudi Arabian consulate in Istanbul to obtain documents for his planned marriage but was not seen exiting the building. He was declared a missing person and was later found dismembered in a suitcase with one of the accomplices of his murder \cite{JamalAssasination}. The investigation conducted by the UNHR concluded that the assassination was a premeditated extrajudicial execution \cite{JamalUNHR}. Additionally, further investigation from the journalist group Forbidden Stories found that Pegasus was being used on the phone of Jamal Khashoggi's wife. \cite{JamalPhone}

\subsection{Khadija Ismayilova}
Khadija Ismayilova is an investigative journalist from Azerbaijan. Her investigative focus is the financial corruption in the Azerbaijani government. This has caused her to be blackmailed, imprisoned and banned from travel. However, that is not all, she was also being spied on by the government through Pegasus. What is remarkable about this situation is the emotional impact that it left on her when she was informed about the spyware on her phone. \cite{KhadijaPhone}

\subsection{Klementyna Suchanow}
Klementyna Suchanow is an author, journalist, and women's rights activist. She was another victim of Pegasus spyware given her strong presence in women's rights activism in Poland. What is remarkable about this case is: \say{It’s so funny because, at the time, Polish counter-intelligence was not doing any work to pursue real spies like Pablo Gonzalez, and didn't use Pegasus against him, but instead they used Pegasus against regular Polish citizens who were just trying to loudly demand their rights,}\cite{KlementynaPhone}. This shows that the government wasn't really using Pegasus for it's "intended" purpose exclusively and the NSO Group was not really marshalling the usage of their product.

\section{Response to the investigation}
The cases mentioned prior and several more came out to light when the non-profits Forbidden Stories and Amnesty International received a list of over 50,000 phone numbers from Pegasus infected phones. This brought the existence of Pegasus to the public and brought major controversy to the NSO Group.

% NOTE: Left off here, finish this part and start on technical for tomorrow's peer review
